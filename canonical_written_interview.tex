\documentclass{exam}
\usepackage{hyperref}
\begin{document}

\begin{center}
\fbox{\fbox{\parbox{5.5in}{\centering
Written interview for Software Engineer - Python and K8s at Canonical.}}}
\end{center}

\vspace{5mm}
\makebox[0.75\textwidth]{Name: Boussejra Amir}

\vspace{5mm}
\makebox[0.75\textwidth]{Reviewer: Varshi Gupta}

\section*{Education}

\begin{questions}
\question How did you fare in high school mathematics, physical sciences, and computing?
\\~\\
I was what people would consider a top student in all sciences related field. 
In high school I was studying mathematics/physics and biology and would always get what could be called a grade A+ (between 18/20 and 20/20 in french notation).

\question What sort of high school student were you? What would your high school peers remember you for, if we asked them?
\\~\\
I was the kind of student that always got top grades, I was still very extrovert and my high school peers would remember me to be the kind of guy 
that was top of the school and at the same time doing push-ups competitions with my group of friends outside of school.

\question In languages and the arts at high school, what were your strongest subjects and how did you rank in those among your school peers?
\\~\\
I have always been very good at languages, both English and Spanish. I was top of the class.
But my Spanish level lowered quite a bit due to not using it anymore in my everyday life.


\question Please outline some high school achievements considered exceptional by peers and staff members.
\\~\\
I finished for the baccalauréat (high school end of study exam) top of my high school while being one of the only students that worked all week-ends.
(My parents had a pub, and I was helping 10 hours per week during the week-end).

\question Which degree and university did you choose, and why?
\\~\\
I choose Mathematics and Theoretical physics at first and later on specialized in Applied Mathematics and Computer Science at 
École des Mines de Saint-Étienne. I always loved science and was very good at them. And while I do more applied things, I do consider
a good training for life to having done in depth theoretical sciences (Measure Theory, Topology, Formal logic, Quantum physics, Statistical Thermodynamics).

\question What did you enjoy most about your time at university?
\\~\\
I was learning much and had time to give tutoring in math and physics as well as doing much sports. I did my 200kgs deadlift during those years !

\question Which university courses did you perform best at? How did you rank in your degree?
\\~\\
I always performed best in Math and Physics, always ranking in the top 5.

\question Outside of class, what were your interests and where did you spend your time?
\\~\\
My two main interest were music, I did guitar everyday those days and mostly sports in my time-off. 
I was hitting the gym 5 times a week and was managing our school's gym as a benevolent trainer.

\question What did you achieve at university that you consider exceptional?
\\~\\
I consider that being successful at university doing a double master's degree 
without a supplementary year while managing tutoring students for 8 hours a week and doing part-time (1 day/week work time) freelance work (Epigno on my CV)
was fairly out of common. 
\end{questions}

\section*{Career Development}

\begin{questions}

\question Please describe any experience as a professional software engineer, working on commercial products.
\\~\\
I will describe my two main experiences without counting internships. At a startup and at a more tradictional company. \\~\\
Epigno (Startup): My main work topic was implementing a doctor’s schedule optimization program. Work was really interesting, as 
I needed to take into account many labor law regulations specific to japanese health industry, mixing my knowledge in linear optimization and computer science
to my PLM giving me concrete indications on what regulations constrain the optimization problem. I loved the fact that my work's goal 
was to help hospital better manage their resources and automatize shift planning. 
I think automatizing tedious work is the best use case of applied mathematics and computer science.
Company was remote first, had a very good culture and the two months I spend in Japan with the team was really one of the best moments of my life (from a human relationship point of view).
\\~\\
Viavi Solutions:
My main work topic was introducing AI into their company. Their main focus is doing fiber testing products, and the goal 
is the analysis of optical events in an OTDR trace. Everything was to be done there, apart from a few 100 lines python script doing linear regression, nothing was done to introduce ML in their analysis.
Project was vast and full of freedom because I was the only one with knowledge on what we could do.
Thus it was structured in 5 steps:
\begin{itemize}
\item Creating a DB to evaluate their signal processing algorithms
\item Creating a Python API to parse their proprietary file format
\item Prototyping a ML MVP that improved their signal processing algorithms
\item Standardizing the ML dev environment (Data processing pipelines with Kedro, MLFlow model registry...)
\item Export the produced models to \href{https://fr.wikipedia.org/wiki/Open_Neural_Network_Exchange}{ONNX} to exploit in a 30 year old C/C++ codebase alongside legacy algorithms.
\end{itemize}
Overall, the work was motivating because everything was to be done, but there was a lack of leadership which lead me to be the sole decidor at many times.
One of the successes was selling a new software license for those ML algorithms which lead to a big business use case with ASML.
\question Please describe any experience as a software engineer working on internal company projects rather than commercial products.
\\~\\
My main experiences on internal company projects would be the following ones:
\begin{itemize}
    \item Epigno: Upgrade a codebase’s frontend from Node 8 to 16, backend from Python 3.4 to 3.9
    \item Epigno: Migrate to \href{https://python-poetry.org/}{Poetry} for python env management
    \item Viavi: Python parser for proprietary sor file format
    \item Viavi: Dashboard driven web-app (streamlit) to qualify statistically their signal processing algorithms
    \item Viavi: Docker environments for C/C++ 30 year old codebase to freeze release build
\end{itemize}

\question Describe your skill in your preferred development language, and how you attained it.
\\~\\
My favorite development language is \href{https://www.rust-lang.org/fr}{Rust} !
This language was introduced to me when I was 20 year old by my 6 year older big brother who did is computer science thesis using this language.
Having time, I was able to build theoretical knowledge by reading fully the \href{https://doc.rust-lang.org/book/}{Rust Book} and later on I improved
by working on the commercial product of the doctor’s shift optimizer at Epigno, developping with a senior engineer recurrent reviews.
Another point that helped me deepen my understanding of the language was teaching it, developing with a physicist non-programmer friend
 a \href{https://github.com/Aboussejra/relativistic_ray_tracing}{relativistic ray-tracer}. And lastly, I think continuously keeping up to date 
 with the \href{https://news.ycombinator.com/news}{Hacker News} Rust related posts helped me improve.

\question What are your strengths as a software engineer?
\\~\\
I am a scientist first before being a software engineer, that leads me to be a rigorous person which always advance my points with rigorous and sourced justifications.
This leads me to be a clean code enthusiast, I love people who want to improve, I will always help those people and never drag them down.
My motto traduced in english would be "Together to do better than alone".
Adding to those technical and interpersonal skills I am a continuous learner, science never stops !

\question What experience do you have with Linux system administration? What is the largest group of servers you have helped operate?
\\~\\
In terms of Linux system administration, my experience primarily stems from managing
my personal system and a big virtual machine I used at Viavi for ML computing. 
While my direct experience might be limited to these environments, I've gained proficiency in tasks such as package management, system monitoring, and basic troubleshooting within a Linux environment.

\question What experience do you have with site reliability engineering, keeping production services online and available?
\\~\\
Again my experience is quite limited in this domain,
The only thing I have contributed to would be  through my role in updating at Epigno the doctor’s schedule optimization microservice. 
I've gained  some experience in implementing updates and push them to production to improve performance of the microservice.
\end{questions}

\section*{Experience}

\begin{questions}

\question Describe your level of experience in Python, and how you have attained it.
\\~\\
My Python level has been a gradual climb, scaling up with each adventure as I always had at least some Python at all my jobs. I would summarise it this way:
\begin{itemize}
    \item 18 years old, not understanding what an interpreted language is and working in a school installed IDE implementing mathy things such as
differential equation solving, graph coloration
    \item 21 years old, first internship, discovering backend development, Flask/FastApi, Venv creation, Python dependency management, formatting, linting
    \item 22 years old, Part-time work. Discovering real backend development, python dependency hell and upgrade hell if not kept up to date regularly. Introduced to more best practices, CI/CD, unit and integration test
    \item 25 years old, becoming a real Pythonista. Trying to improve the workflow everywhere, pushing linting and static type checking everywhere alongside data validation (Glory to \href{https://docs.pydantic.dev/latest/}{Pydantic}). Encouraging hardware guys who only does some Excel and basic scripting on windows without virtual envs to begin managing virtual environments.
\end{itemize}

\question When did you start working on Linux? Describe your level of experience as a user \& developer on Linux.
\\~\\
I started working on Linux with my first internships. I was a user from my 12 years old 
using Ubuntu (first DVD was engraved by my older brother), but with a Windows dual-boot at the time to be able to play some video-games 
(At the time I was not a huge linux tinkerer and playing with Wine was complicated, now I play on SteamOS).
I always use linux developping or for regular uses (mainly \href{https://pop.system76.com/}{Pop!\_OS} and \href{https://www.debian.org/index.fr.html}{debian})
\question Describe your experience with container technologies (Docker, LXD, Kubernetes, etc).
\\~\\
I never had a full time job managing container like a devops could. But I often did some docker and docker compose (no kubernetes) when need was there.
Mainly consisting of adding a container microservice in an existing compose orchestration and doing some isolated Dockerfile to yield a reproducible test environment.
\question How do you prefer to drive documentation for your products?
\\~\\
I prefer managing doc as code. I think that when doc is not managed that way it often becomes obsolete fast. 
That means I like my doc to be under git (like my resume is), in plain text markup. 
If you need extensive documentation, I like how things are done using tools like \href{https://github.com/rust-lang/mdBook}{mdBook} or \href{https://www.gitbook.com/}{GitBook}
\question How do you think about quality in your products?
\\~\\
I like to have a quality-first approach in software products. 
For example I try to minimize technical debts by often updating the toolchains and libraries that I am using. I learned the hard way how tough it can be to migrate a library which had 10 years of API changes.
I think working as a team with a group of excellent technical people and having enough time permits to produce quality products.
\question Describe a case where it was very difficult to test code you were writing, but you found a reliable way to do it.
\\~\\
At Viavi Solutions, I needed to add my ML algorithms to an existing legacy solution after MVP was developped.
And we had a problem in evaluating measurement algorithms on the OTDR traces. An existing solution was to load sequentially 
files that were written on a Hard Disk Drive to an embedded product that then run the measurement procedure on each file, 
and lastly we would read a json report of what was displayed on screen in the results. 
A big problem was that what we saw displayed could change depending on the config of the embedded product (units like miles/meters, filtering of some files due to client configuration).
Thus we were obligated to make sure we put the product manually in a good configuration before running tests, and software updates of what was in the report could cause variations of the output parsed.
For this reason we developed a Python parser for the proprietary files (they do not depend on the config) to ensure consistent input and a new tool to load the files in a product,
get back measured file and be able to test the measures with a consistent output not depending on the configuration. 
At the same time work was done to be able to interface it with a x86 build of the minimal measurement app that were on the product and not the full tooling.
\question What would you like to achieve in career development and skills development?
\\~\\
My main goal is to be able to constantly learn, my biggest fear is entering an environment where every day is the same and there is no technological thrill.
\end{questions}

\section*{Context}

\begin{questions}

\question Are you involved in open source software?
\\~\\
Not really, the only contribution I could show would be a regression problem in the rust language I was able to isolate and report.
\question Describe any significant contributions to open source (with links where possible), if any.
\\~\\
The mentioned problem is the following one:
https://github.com/rust-lang/rust/issues/103141
\question Why do you most want to work for Canonical?
\\~\\
I have been told Canonical is full of excellent engineers and it will permit working on interesting open-source projects. Those are my two main motivations.
\question Which other companies are building the sort of products you would like to work on?
\\~\\
A company I would really like to work with is \href{https://huggingface.co/}{Hugging Face}, they have data engineering tooling that is 
rusty and build a great ML platform to help developers. The intersection of my favourite language and one of my favorite topics would make this company one I find really interesting.
\question Which companies have the most interesting approach to devops and site reliability engineering?
\\~\\
I do not have much knowledge here I must admit, but I would say Amazon with AWS must be very good at this topic. 
I think that people at Netflix or other kind of streaming services are probably very good at ensuring good devops and service reliability.
\question What do you think could raise the bar for site reliability engineering, globally?
\\~\\
I think having some penalty when something is not reliable could push people to put reliability as a first-class citizen.
I think often big deals can be concluded in a good demo and later on the delivered product is not as good but is already sold.
(Example of some services in public French company like \href{https://www.sncf.com/fr}{SNCF} for example, which is unreliable and buggy as hell).
\question What do you think Canonical needs to improve in its engineering and products?
\\~\\
To be honest, I think talking about what a company needs to improve without having worked in it is a bit of a shallow thinking.
\question Who do you think are key competitors to Canonical? How do you think Canonical should plan to win that race?
\\~\\
I think RedHat and SUSSE are the main competitors to Canonical.
Once again I would rather not give a shallow thinking on what should Canonical do not knowing the business that is related so I will refrain from answering.
\end{questions}


\end{document}
